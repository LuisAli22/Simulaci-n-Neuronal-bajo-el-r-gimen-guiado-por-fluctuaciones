Durante la estimulación sensorial\footnote{Un estímulo sensorial es un estímulo a los sentidos de una persona. Los más comunes son estímulos visuales, auditivos, el olfato, el tacto, etc.}, las neuronas de la corteza visual se someten a un bombardeo sináptico y los potenciales de acción son principalmente el resultado de fluctuaciones en el potencial de membrana. Para entender las propiedades de las respuestas de las neuronas operando en este régimen, se estudia el modelo de una neurona con entradas sinápticas representadas por cambios en la conductancia de membrana. En la investigación \cite{Kuhn2345} se demuestra que con el incremento simultáneo de excitación e inhibición, la tasa de disparo primero aumenta, llega a un máximo y después decae, a mayor tasa de entrada (a mayor bombardeo). Por lo tanto, la comodulación de excitación e inhibición no brinda una manera sencilla de controlar la tasa de disparo neuronal. La conductancia inducida juega un rol principal en este efecto:
\begin{itemize}
    \item \textit{Disminuye la tasa de disparo} desviando las fluctuaciones del potencial de membrana.
    \item \textit{Incrementa la tasa de disparo} al reducir la constante de tiempo de la membrana.
\end{itemize}
En el paper \cite{Kuhn2345}, primero se investiga la respuesta neuronal frente a una entrada balanceada. Se analiza como cambian con el nivel de bombardeo, la integración de eventos sinápticos individuales, las fluctuaciones del potencial de membrana y la tasa de disparo. Como segundo paso, se relaja la condición de entrada balanceada y se estudia la respuesta neuronal a niveles arbitrarios de excitación e inhibición.
